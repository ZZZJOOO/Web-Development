\documentclass{article}
\begin{document}

\section{Introduction to Front End Development}

\subsection*{Objective}
\begin{itemize}
\item Set up Environment
\item Difference between front-end and back-end
\item Roles of HTML, CSS and JavaScript
\end{itemize}



\subsection{Setting up Front End Developer Environment}
\begin{itemize}
	\item Sublime Text 3
	\item Google Chrome
\end{itemize}

\subsection{Internet Basics}
\begin{itemize}
	\item How Internet works?
	\item What is the difference between front-end and back-end?
	\item How to view the HTML file on any website?
\end{itemize}

\paragraph{Internet}
HTTP request sent from a user will figure out a way to get to the correct server. The server then send the desired web page into data package back to the user through the internet. How the server response here actually depends on the design of the back-end.

\paragraph{Concept Comparison}
\begin{itemize}
	\item \textbf{Front-End} is all the things you can see in a web browser: HTML, CSS, JavaScript
	\item \textbf{Back-End} is everything else, like managing the databases(if it is on dynamic web page)
\end{itemize}

\paragraph{HTML} defines the structure of the web page: text, image etc.

\paragraph{CSS} defines the style: color on text or image boarder etc.

\paragraph{JavaScript} defines the interactivity: how user can interact with the web page.

\newpage

\section{Intro to HTML}
\subsection*{Objectives}
\begin{itemize}
	\item HTML basics
	\item Write a simple HTML page
\end{itemize}

\subsection{HTML Basics}
\begin{itemize}
	\item History of HTML
	\item General Rule
\end{itemize}

\paragraph{History} Researcher would like to share scientific and technical documents.

\paragraph{General Rule}
$<tagname>$  Context  $</tagname>$

\end{document}








































